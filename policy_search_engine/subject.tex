% This is LLNCS.DEM the demonstration file of
% the LaTeX macro package from Springer-Verlag
% for Lecture Notes in Computer Science,
% version 2.4 for LaTeX2e as of 16. April 2010
%
\documentclass{llncs}
%
\usepackage{makeidx}  % allows for indexgeneration
\usepackage{hyperref}
\hypersetup{
  colorlinks=true,
  linkcolor=black,
  citecolor=blue,
  urlcolor=blue
}
%
\begin{document}
%
\mainmatter              % start of the contributions
%
\title{Policy aware search engine}

%
\author{Benjaming Moreau\inst{2}}
%
\institute{
LS2N -- University of Nantes\\
\email{firstname.lastname@univ-nantes.fr}
\and
OpenDataSoft \email{firstname.lastname@univ-nantes.fr}}

\maketitle              % typeset the title of the contribution

% \begin{abstract}
% The abstract should summarize the contents of the paper
% using at least 70 and at most 150 words. It will be set in 9-point
% font size and be inset 1.0 cm from the right and left margins.
% There will be two blank lines before and after the Abstract. \dots
% \end{abstract}
%
\keywords{Semantic Web $\cdot$ Data usage $\cdot$ Policies $\cdot$ Search Engine}

\section*{Background and motivation}

Semantic Web expose thousands of datasets in a way that help data sharing and data analysis. Privacy policies are often attached to these datasets. These policies are described in RDF using specific vocabularies\cite{soto2015policies} and describe how to use data, what is permitted, obliged or prohibited. 

On the one hand, data publisher are able to publish datasets with policies explicitely described in a machine readable format.  On the other hand, data consumers ( i.e human or machine ) are not able to express their profile ( i.e what they will do with the data ) and thus, cant discover only dataset matching their usage. 

\section*{Objectives}
Given a group of dataset in RDF format, the objectif of this supervised work will be to:
\begin{itemize}
	\item Express datasets policies with an ontology.
	\item Find a way to describe data consumer's profile.
	\item Implement a policy aware search engine that help data consumers to find dataset corresponding to their profile.
	\item Evaluate your approach with differents metrics ( e.g completeness, execution time, etc. )
\end{itemize}
\section*{Remarks}
You will have to find an elegant way to implement the search engine in order to minimize total search execution time and memory usage.

You can find exemple\footnote{http://data.europa.eu/euodp/en/linked-data} of RDF search engine dataset on the web.

\bibliographystyle{splncs03}
\bibliography{subject}
\end{document}
